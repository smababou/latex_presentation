\section{Datenbanken}
\begin{frame}
\frametitle{Relation}
\begin{block}{Def.: Relation}
Eine Menge $R$ heißt $n$-stellige (auch $n$-äre) Relation über Mengen $A_1,A_2,\ldots,A_n$, falls
\begin{equation*}
R \subseteq \{ (x_1,x_2,\ldots, x_n) | x_1 \in A_1 \text{ und } x_2\in A_2 \text{ und  } \ldots \text{ und } x_n \in A_n \}
\end{equation*}
gilt.
\end{block}
Anmerkungen:\\
\begin{itemize}
\item Eine Relation $R$ ist eine Menge von Tupeln
\item Die $i$-te Komponente eines Tupels ist ein Element von $A_i$
\end{itemize}
\end{frame}
\begin{frame}
\frametitle{Beispiel zur Relation}
Es sei
\begin{itemize}
\item $A_1 = \{ \text{Nudeln, Ketchup, Eier }\}$
\item $A_2 = \mathbb{N}_0$
\item $A_3 = \mathbb{R}_{\geq 0}$
\end{itemize}
\pause
Eine Relation $R$ über $A_1,A_2,A_3$ ist etwa
\begin{equation*}
R = \{(\text{Nudeln}, 3, 0.9),(\text{Ketchup}, 1, 1.50),(\text{Eier}, 2, 1.99) \}.
\end{equation*}
Hier ist $R$ eine Bestandsliste der Lebensmittel mit Einkaufspreisen in einer Studenten-WG, z.B. sind drei Nudelpackungen zu einem Einkaufspreis von je 0,90 € vorhanden.
\end{frame}
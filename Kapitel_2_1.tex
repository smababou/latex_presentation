	\begin{frame}
		\frametitle{Was sind Daten?}
		\pause
		Je nach Fachbereich werden verschiedene, teilweise sehr ähnliche, Definitionen verwendet. Hier orientieren wir uns an folgender Definition:
		\pause
		\begin{block}{Daten}
		''reinterpretable representation of information in a formalized manner suitable for communication, interpretation, or processing''\footnote{ISO/IEC 2382:2015}
		\end{block}
		\pause
		\begin{itemize}
		\item Daten sind hier also eine (digitale) formalisierte Darstellung von Informationen
		\pause
		\item Die Informationen sind wieder interpretierbar
		\pause
		\item Die formalisierte Darstellung ist so geschaffen, dass die Informationen für Kommunikation, Interpretation und Auswertung verwendet werden können 
		\end{itemize}
	\end{frame}



\begin{frame}
\frametitle{Anwendungen für (komplexe) Datenanalysen}
\begin{itemize}[<+->]
\item Sortierung von Briefen mittels automatisierter Erkennung von handschriftlichen Adressen
\item Erkennung von Gegenständen für automatisiertes Picking
\item Predictive Maintenance
\item Routenplanung
\item Autonome Roboter und Fahrzeuge
\item Analyse von Kundenverhalten
\item Prognose von Durchlaufzeiten, Mitarbeiterbedarf,...
\item Anomalieerkennung in der Fertigung
\end{itemize}
\end{frame}